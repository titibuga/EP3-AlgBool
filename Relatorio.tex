\documentclass[11pt]{article}
\usepackage[brazilian]{babel}
\usepackage[utf8]{inputenc} %Deixa eu colocar letras com ascentos
\usepackage[T1]{fontenc}
\usepackage{amsmath}
\usepackage{color}
\usepackage{tikz}
\usetikzlibrary{arrows, decorations.pathmorphing,backgrounds, positioning, fit, petri}



\title{Exercício Programa 3}

\begin{document}

\maketitle

\section{Introduç\~ao}

Este relatório tem como objetivo dar uma explicação das principais funcionalidades do circuito feito nesse exercício programa. Assim, esse relatório pode ser de grande ajuda na hora de entender o EP.


\subsection*{Integrantes}

\begin{itemize}

\item Victor Sanches Portella - Nº USP: 7991152

\item Mateus Barros Rodrigues - Nº USP: 7991037

\item Gervásio Santos - Nº USP:

\item Vinícius Vendramini - Nº USP: 

\end{itemize}

\section{Circuito}

Como a \textbf{Memória RAM}, o \textbf{IR}  e o \textbf{ACC} são circuitos já implementados pelo Logisim, explicaremos a funcionalidade do \textbf{PC}, do \textbf{Controle} e como esses circuitos são integrados. 

\subsection{PC (Program Counter)}

Para fazer o PC, usamos 8 flip-flops JK, todos inicialmente em 0. Quando um pulso é mandado para a entrada \emph{Counter} do circuito, somamos 1 no contador, considerando que os 8 flip-flops forma um número de 8 bits.

Quando a entrada \emph{SETTER} do PC está alta, o valor dos flip-flops (ou seja, do contador em si) será igual ao valor de 8 bits da entrada 	\emph{Number} do circuito. Em \emph{OUT} temos o valor do contador do PC.

\subsection{Controle}

Para fazer o controle, fizemos um circuito sequencial com quatro estados:

\begin{itemize}

\item \textbf{Estado 0:} Tranfere o que está no endereço apontado pelo PC na Memória RAM para o IR.

\item \textbf{Estado 1:} Tranfere o conteúdo do IR para dentro do controlador.

\item \textbf{Estado 2:} Interpretação do comando.

\item \textbf{Estado 3:} Soma-se em 1 o valor do PC.

\end{itemize}

Sendo que esses estados seguem o seguinte diagrama:
\begin{figure}[!hbtp]
\centering


\begin{tikzpicture}[->,>=stealth',shorten >=1pt,estado/.style={circle, draw=black, fill=blue!20, thick}]

\node[estado]  (e0)  at (0,0)  {e0};
	
\node[estado]  (e1)  at (5,0)  {e1};
	
\node[estado]  (e2)  [below=of e1]  {e2};
     	
\node[estado]  (e3)  [below=of e0]   {e3};
	

\path
(e0) edge [loop above] node {0/0} (e0)
	
(e1) 	 edge [loop above] node {0/0} (e1)
    		 edge [<-, bend right] node [above]{1/0} (e0)
    
(e2)     	edge[loop below] node {0/1} (e2)
 	     	edge[<-, bend right] node [right] {1/1} (e1)
	
(e3)		 edge [<-, bend right] node [below] {1/1} (e2)
		 edge [->, bend left] node [left] {1/0} (e0)
		 edge[loop below] node {0/1} (e3);


\end{tikzpicture}


\end{figure}	

Para a interpretação dos comandos, verificamos o byte armazenado no IR relativo a instrução. Dependendo de qual instrução é lida, mandamos os sinais necessários para as outras peças:

\begin{itemize}

\item \textbf{{\color{red}LDA:}} Deixa a memória no modo \emph{OUTPUT},  e ACC no modo \emph{INPUT}.

\item \textbf{\color{red}STA:} Colcoa a memória no modo de \emph{INPUT}, e o ACC no modo de \emph{OUTPUT}.

\item \textbf{\color{red}JMP:} Coloca o PC em modo de \emph{SET}, fazendo ele receber o endereço guardado no IR.

\item \textbf{\color{red}NOP:} Não faz nada durante a fase 2 do controle.

\item \textbf{\color{red}STOP:} Faz com que o clock pare de ser passado para o controle, parando todo o sistema.

\end{itemize}





\end{document}